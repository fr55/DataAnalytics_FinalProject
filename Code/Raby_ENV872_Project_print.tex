\documentclass[12pt,]{article}
\usepackage{lmodern}
\usepackage{amssymb,amsmath}
\usepackage{ifxetex,ifluatex}
\usepackage{fixltx2e} % provides \textsubscript
\ifnum 0\ifxetex 1\fi\ifluatex 1\fi=0 % if pdftex
  \usepackage[T1]{fontenc}
  \usepackage[utf8]{inputenc}
\else % if luatex or xelatex
  \ifxetex
    \usepackage{mathspec}
  \else
    \usepackage{fontspec}
  \fi
  \defaultfontfeatures{Ligatures=TeX,Scale=MatchLowercase}
    \setmainfont[]{Times New Roman}
\fi
% use upquote if available, for straight quotes in verbatim environments
\IfFileExists{upquote.sty}{\usepackage{upquote}}{}
% use microtype if available
\IfFileExists{microtype.sty}{%
\usepackage{microtype}
\UseMicrotypeSet[protrusion]{basicmath} % disable protrusion for tt fonts
}{}
\usepackage[margin=2.54cm]{geometry}
\usepackage{hyperref}
\hypersetup{unicode=true,
            pdftitle={Analysis of the effect of PM10 Concentration, Temperature, Geographic Location, Population, and distance to Electricity-Generation Combustion Points on the concentration of fine particulate matter (PM2.5) in North Carolina during the year 2018.},
            pdfauthor={Felipe Raby Amadori},
            pdfborder={0 0 0},
            breaklinks=true}
\urlstyle{same}  % don't use monospace font for urls
\usepackage{graphicx,grffile}
\makeatletter
\def\maxwidth{\ifdim\Gin@nat@width>\linewidth\linewidth\else\Gin@nat@width\fi}
\def\maxheight{\ifdim\Gin@nat@height>\textheight\textheight\else\Gin@nat@height\fi}
\makeatother
% Scale images if necessary, so that they will not overflow the page
% margins by default, and it is still possible to overwrite the defaults
% using explicit options in \includegraphics[width, height, ...]{}
\setkeys{Gin}{width=\maxwidth,height=\maxheight,keepaspectratio}
\IfFileExists{parskip.sty}{%
\usepackage{parskip}
}{% else
\setlength{\parindent}{0pt}
\setlength{\parskip}{6pt plus 2pt minus 1pt}
}
\setlength{\emergencystretch}{3em}  % prevent overfull lines
\providecommand{\tightlist}{%
  \setlength{\itemsep}{0pt}\setlength{\parskip}{0pt}}
\setcounter{secnumdepth}{5}
% Redefines (sub)paragraphs to behave more like sections
\ifx\paragraph\undefined\else
\let\oldparagraph\paragraph
\renewcommand{\paragraph}[1]{\oldparagraph{#1}\mbox{}}
\fi
\ifx\subparagraph\undefined\else
\let\oldsubparagraph\subparagraph
\renewcommand{\subparagraph}[1]{\oldsubparagraph{#1}\mbox{}}
\fi

%%% Use protect on footnotes to avoid problems with footnotes in titles
\let\rmarkdownfootnote\footnote%
\def\footnote{\protect\rmarkdownfootnote}

%%% Change title format to be more compact
\usepackage{titling}

% Create subtitle command for use in maketitle
\newcommand{\subtitle}[1]{
  \posttitle{
    \begin{center}\large#1\end{center}
    }
}

\setlength{\droptitle}{-2em}

  \title{Analysis of the effect of PM10 Concentration, Temperature, Geographic
Location, Population, and distance to Electricity-Generation Combustion
Points on the concentration of fine particulate matter (PM2.5) in North
Carolina during the year 2018.}
    \pretitle{\vspace{\droptitle}\centering\huge}
  \posttitle{\par}
  \subtitle{\url{https://github.com/fr55/DataAnalytics_FinalProject}}
  \author{Felipe Raby Amadori}
    \preauthor{\centering\large\emph}
  \postauthor{\par}
    \date{}
    \predate{}\postdate{}
  

\begin{document}
\maketitle
\begin{abstract}
Experimental overview. This section should be no longer than 250 words.
\end{abstract}

\newpage

\tableofcontents  \newpage
\listoftables  \newpage
\listoffigures  \newpage

\section{Research Question and
Rationale}\label{research-question-and-rationale}

Nowadays air pollution is one of the most relevant health issues in the
world. It refers to the contamination of the air by chemicals,
biological materials, and other types of pollutants that are harmful to
human health. To solve the problem of air pollution, it's necessary to
understand the problem, what are the causes, and search for solutions
based on the findings.

Particulate matter with a diameter of less than 2.5 micrometers is
called PM2.5, and it is a extremely harmful air pollutant because it
consists of particles with diameters that are less than or equal to 2.5
microns in size, which can get deeply into the lung, and ultimately
impair lung function.

This study focus on trying to understand how PM2.5 concentration in
North Carolina vary with temperature, PM10 concentration, zoning
(piedmont, coastal, mountain), population, elevation, and distance to
combustion points for electricity generation. This last variable was
included because according to the EPA combustion for electricity
generation is the major point-source sector for PM2.5 in the USA (EPA,
2019).

The research question is: What are the effects of temperature, PM10
concentration, zoning (piedmont, coastal, mountain), population,
elevation, distance to combustion points for electricity generation, in
PM2.5 concentrations within North Carolina in the year 2018?

\newpage

\section{Dataset Information}\label{dataset-information}

For the analysis the following datasets were considered:

\subsection{EPA PM2.5 Dataset}\label{epa-pm2.5-dataset}

This dataset contains data from air quality monitoring of PM2.5 in North
Carolina in 2018, and it was obtained using the Download Daily Data Tool
in the United States Environmental Protection Agency (EPA) webpage
\url{https://www.epa.gov/outdoor-air-quality-data/download-daily-data}
where the options showed in Table \ref{tab:tab1} were selected:

\begin{table}[ht]
\centering
\begin{tabular}{ll}
  \hline
Option & Selection \\ 
  \hline
Pollutant & PM2.5 \\ 
  Year & 2018 \\ 
  Geographic Area & North Carolina \\ 
  Monitor Site & All Sites \\ 
  Download & Download CSV (spreadsheet) \\ 
   \hline
\end{tabular}
\caption{Selections} 
\label{tab:tab1}
\end{table}

The downloaded file was saved in the project folder path ./Data/Raw/ as
EPAair\_PM25\_NC2018\_raw.csv on 2019-03-31.

\subsubsection{Data Content Information}\label{data-content-information}

The dataset contains daily mean PM2.5 concentration in ug/m3 in 2018.
Data from 24 stations in 21 different counties of North Carolina with
their location in NAD83 lat/long coordinates.

The dataset contains 19 columns, which are shown in Table
\ref{tab:tab3}. Column names without description are self-explanatory.

\begin{table}[ht]
\centering
\begin{tabular}{p{2.5in}p{3.5in}}
  \hline
Column & Description \\ 
  \hline
Date & mm/dd/YY \\ 
  Source & AQS (Air Quality System) \\ 
  Site ID & A unique number identifying the site. \\ 
  POC & “Parameter Occurrence Code”, distinguishes different instruments that measure the same parameter at the same site. \\ 
  Daily Mean PM2.5 Concentration &  \\ 
  Units & Concentration Units \\ 
  DAILY\_AQI\_VALUE & AQI = Air quality index \\ 
  Site Name &  \\ 
  DAILY\_OBS\_COUNT &  \\ 
  PERCENT\_COMPLETE &  \\ 
  AQS\_PARAMETER\_CODE &  \\ 
  AQS\_PARAMETER\_DESC &  \\ 
  CBSA\_CODE &  \\ 
  CBSA\_NAME &  \\ 
  STATE\_CODE &  \\ 
  COUNTY CODE & A unique number identifying the County. \\ 
  COUNTY &  \\ 
  SITE\_LATITUDE & NAD83 \\ 
  SITE\_LONGITUDE & NAD83 \\ 
   \hline
\end{tabular}
\caption{Dataset content} 
\label{tab:tab3}
\end{table}

\subsection{EPA PM10 Dataset}\label{epa-pm10-dataset}

This dataset contains data from air quality monitoring of PM10 in North
Carolina in 2018, and it was obtained using the Download Daily Data Tool
in the United States Environmental Protection Agency (EPA) webpage
\url{https://www.epa.gov/outdoor-air-quality-data/download-daily-data}
where the options showed in Table \ref{tab:tab4} were selected:

\begin{table}[ht]
\centering
\begin{tabular}{ll}
  \hline
Option & Selection \\ 
  \hline
Pollutant & PM10 \\ 
  Year & 2018 \\ 
  Geographic Area & North Carolina \\ 
  Monitor Site & All Sites \\ 
  Download & Download CSV (spreadsheet) \\ 
   \hline
\end{tabular}
\caption{Selections} 
\label{tab:tab4}
\end{table}


\end{document}
